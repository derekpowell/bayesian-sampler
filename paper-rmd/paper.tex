\documentclass{article}

\usepackage{arxiv}

\usepackage[utf8]{inputenc} % allow utf-8 input
\usepackage[T1]{fontenc}    % use 8-bit T1 fonts
\usepackage{lmodern}        % https://github.com/rstudio/rticles/issues/343
\usepackage{hyperref}       % hyperlinks
\usepackage{url}            % simple URL typesetting
\usepackage{booktabs}       % professional-quality tables
\usepackage{amsfonts}       % blackboard math symbols
\usepackage{nicefrac}       % compact symbols for 1/2, etc.
\usepackage{microtype}      % microtypography
\usepackage{graphicx}

\title{A template for the \emph{arxiv} style}

\author{
    Derek Powell
   \\
    School of Social and Behavioral Sciences \\
    Arizona State University \\
  Phoenix, AZ \\
  \texttt{\href{mailto:dmpowell@asu.edu}{\nolinkurl{dmpowell@asu.edu}}} \\
  }


% tightlist command for lists without linebreak
\providecommand{\tightlist}{%
  \setlength{\itemsep}{0pt}\setlength{\parskip}{0pt}}


% Pandoc citation processing
\newlength{\cslhangindent}
\setlength{\cslhangindent}{1.5em}
\newlength{\csllabelwidth}
\setlength{\csllabelwidth}{3em}
\newlength{\cslentryspacingunit} % times entry-spacing
\setlength{\cslentryspacingunit}{\parskip}
% for Pandoc 2.8 to 2.10.1
\newenvironment{cslreferences}%
  {}%
  {\par}
% For Pandoc 2.11+
\newenvironment{CSLReferences}[2] % #1 hanging-ident, #2 entry spacing
 {% don't indent paragraphs
  \setlength{\parindent}{0pt}
  % turn on hanging indent if param 1 is 1
  \ifodd #1
  \let\oldpar\par
  \def\par{\hangindent=\cslhangindent\oldpar}
  \fi
  % set entry spacing
  \setlength{\parskip}{#2\cslentryspacingunit}
 }%
 {}
\usepackage{calc}
\newcommand{\CSLBlock}[1]{#1\hfill\break}
\newcommand{\CSLLeftMargin}[1]{\parbox[t]{\csllabelwidth}{#1}}
\newcommand{\CSLRightInline}[1]{\parbox[t]{\linewidth - \csllabelwidth}{#1}\break}
\newcommand{\CSLIndent}[1]{\hspace{\cslhangindent}#1}

\usepackage{booktabs}
\usepackage{longtable}
\usepackage{array}
\usepackage{multirow}
\usepackage{wrapfig}
\usepackage{float}
\usepackage{colortbl}
\usepackage{pdflscape}
\usepackage{tabu}
\usepackage{threeparttable}
\usepackage{threeparttablex}
\usepackage[normalem]{ulem}
\usepackage{makecell}
\usepackage{xcolor}
\begin{document}
\maketitle


\begin{abstract}
Enter the text of your abstract here.
\end{abstract}

\keywords{
    blah
   \and
    blee
   \and
    bloo
   \and
    these are optional and can be removed
  }

\hypertarget{introduction}{%
\section{Introduction}\label{introduction}}

Bayesian theories of cognition have had remarkable successes in
explaining human reasoning and behavior across many domains (big cite).
{[}The core of these theories is that people reason according to
subjective mentally-represented degrees of belief, and they specify how
they should be revised in light of evidence. It is somewhat embarrassing
then that one area where these theories seem to fall down is in
describing human ``beliefs'\,' of the simple and everyday sort, such as
beliefs like''it will rain tomorrow``, ``vaccines are safe,'' or ``this
politician is trustworthy.''

Trouble starts as soon as we seek to measure beliefs. According to
Bayesian theories of cognition and epistemology, the degree to which
people believe in various propositions should reflect subjective mental
probabilities. So asking people to express beliefs in terms of
probability seems only natural.

Unfortunately, people's explicit probability judgments routinely violate
the axioms of probability theory. For example, human probability
judgments often exhibit the ``conjunction fallacy'': people will often
judge the conjunction of two events (e.g.~``Tom Brady likes football and
miniature horses'') as being more probable than one of the events in
isolation (e.g.~``Tom Brady likes miniature horses''), a plain and
flagrant violation of probability theory (cite some examples). Other
demonstrations of the incoherence of probability judgments include
disjunction fallacies (e.g.~XXX), ``unpacking'' effects (e.g.~fox \&
tversky), and a variety of other effects illustrating the incoherence of
human probability judgments {[}cite{]}. Altogether these findings have
led many researchers to abandon the notion that credences are
represented as probabilities.

Recently however, two groups of researchers have proposed theories of
human probability judgments that account for biases and apparent
incoherence in these judgments while maintaining that mental credences
are fundamentally probabilistic (Costello and Watts 2014; Zhu, Sanborn,
and Chater 2020). Both of these theories build on the increasingly
popular notion that a variety of human reasoning tasks are accomplished
by drawing a limited number of samples from probabilistic mental models
(see also Chater et al. 2020; Dasgupta, Schulz, and Gershman 2017).

\hypertarget{two-probabilistic-theories-of-probability-judgment}{%
\subsection{Two probabilistic theories of probability
judgment}\label{two-probabilistic-theories-of-probability-judgment}}

Costello \& Watts (2014, 2016, 2018) proposed a theory of probability
judgment they call the ``Probability Theory plus Noise'' theory (PT+N).
In the PT+N model, mental ``samples'' are drawn from a probabilistic
mental model of events and are then ``read'' with noise, so that some
positive examples will be read as negative and some negative read as
positive. This results in probability judgments that reflect
probabilistic credences perturbed by noise. Under the simplest form of
the PT+N model, the expected value of probability judgments is:

\[E[\hat{P}_{PT+N}(A)] = (1-2d)P(A) + d \]

where \emph{d} represents the probability with which samples will be
misread, or the amount of noise in the judgment process (where by
assumption a maximum of 50\% of samples can be misread, so that d is a
number in the range {[}0, .50{]}). The PT+N theory provides a unified
accounts for a wide variety of biases in probability judgment that were
previously attributed to different types of heuristics (Costello and
Watts 2014, 2016, 2017, 2018).

\hypertarget{refs}{}
\begin{CSLReferences}{1}{0}
\leavevmode\hypertarget{ref-chater.etal2020}{}%
Chater, Nick, Jian-Qiao Zhu, Jake Spicer, Joakim Sundh, Pablo
León-Villagrá, and Adam Sanborn. 2020. {``Probabilistic {Biases Meet}
the {Bayesian Brain}.''} \emph{Current Directions in Psychological
Science} 29 (5): 506--12.
\url{https://doi.org/10.1177/0963721420954801}.

\leavevmode\hypertarget{ref-costello.watts2014}{}%
Costello, Fintan, and Paul Watts. 2014. {``Surprisingly Rational:
{Probability} Theory Plus Noise Explains Biases in Judgment.''}
\emph{Psychological Review} 121 (3): 463--80.
\url{https://doi.org/10.1037/a0037010}.

\leavevmode\hypertarget{ref-costello.watts2016}{}%
---------. 2016. {``People's Conditional Probability Judgments Follow
Probability Theory (Plus Noise).''} \emph{Cognitive Psychology} 89
(September): 106--33.
\url{https://doi.org/10.1016/j.cogpsych.2016.06.006}.

\leavevmode\hypertarget{ref-costello.watts2017}{}%
---------. 2017. {``Explaining {High Conjunction Fallacy Rates}: {The
Probability Theory Plus Noise Account}.''} \emph{Journal of Behavioral
Decision Making} 30 (2): 304--21.
\url{https://doi.org/10.1002/bdm.1936}.

\leavevmode\hypertarget{ref-costello.watts2018}{}%
---------. 2018. {``Invariants in Probabilistic Reasoning.''}
\emph{Cognitive Psychology} 100 (February): 1--16.
\url{https://doi.org/10.1016/j.cogpsych.2017.11.003}.

\leavevmode\hypertarget{ref-dasgupta.etal2017}{}%
Dasgupta, Ishita, Eric Schulz, and Samuel J. Gershman. 2017. {``Where Do
Hypotheses Come From?''} \emph{Cognitive Psychology} 96 (August): 1--25.
\url{https://doi.org/10.1016/j.cogpsych.2017.05.001}.

\leavevmode\hypertarget{ref-zhu.etal2020}{}%
Zhu, Jian-Qiao, Adam N. Sanborn, and Nick Chater. 2020. {``The
{Bayesian} Sampler: {Generic Bayesian} Inference Causes Incoherence in
Human Probability Judgments.''} \emph{Psychological Review} 127 (5):
719--48. \url{https://doi.org/10.1037/rev0000190}.

\end{CSLReferences}

\bibliographystyle{unsrt}
\bibliography{references.bib}


\end{document}
